\documentclass{article}
\usepackage{amsmath,amsthm,amssymb}
\usepackage{mathtext}
\usepackage[T1,T2A]{fontenc}
\usepackage[utf8]{inputenc}
\usepackage[english, russian]{babel}


\title{Быстрая локализация изображений, получаемых некалиброванной камерой}
\author{Елизавета Миронович}
%\date{October 2019}

\begin{document}

\maketitle

\section{Введение}

Задача определения собственного местоположения является одной из важнейших задач робототехники.
Компьютерное зрение может предоставить альтернативу существующим методам, таким как GPS, которая является более точной и повсеместно доступной, а использование обычных камер позволяет создавать дешёвые решения, в отличие от использования более продвинутых оптических систем, например, лидаров.
Полученные алгоритмы также могут быть использованы и в других областях, таких как 3D-реконструкция и дополненная реальность.

В данной работе будет рассматриваться задача локализации относительно облака 3D точек, в котрой используются проекции этих точек на фотографии, сделанной с помощью некалиброванных камер.
Данная задача не является новой, и цель данной работы -- найти наиболее быстрый алгоритм, который можно было бы применять в реальном времени.

\end{document}

